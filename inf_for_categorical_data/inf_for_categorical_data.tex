% Options for packages loaded elsewhere
\PassOptionsToPackage{unicode}{hyperref}
\PassOptionsToPackage{hyphens}{url}
%
\documentclass[
]{article}
\usepackage{amsmath,amssymb}
\usepackage{lmodern}
\usepackage{iftex}
\ifPDFTeX
  \usepackage[T1]{fontenc}
  \usepackage[utf8]{inputenc}
  \usepackage{textcomp} % provide euro and other symbols
\else % if luatex or xetex
  \usepackage{unicode-math}
  \defaultfontfeatures{Scale=MatchLowercase}
  \defaultfontfeatures[\rmfamily]{Ligatures=TeX,Scale=1}
\fi
% Use upquote if available, for straight quotes in verbatim environments
\IfFileExists{upquote.sty}{\usepackage{upquote}}{}
\IfFileExists{microtype.sty}{% use microtype if available
  \usepackage[]{microtype}
  \UseMicrotypeSet[protrusion]{basicmath} % disable protrusion for tt fonts
}{}
\makeatletter
\@ifundefined{KOMAClassName}{% if non-KOMA class
  \IfFileExists{parskip.sty}{%
    \usepackage{parskip}
  }{% else
    \setlength{\parindent}{0pt}
    \setlength{\parskip}{6pt plus 2pt minus 1pt}}
}{% if KOMA class
  \KOMAoptions{parskip=half}}
\makeatother
\usepackage{xcolor}
\IfFileExists{xurl.sty}{\usepackage{xurl}}{} % add URL line breaks if available
\IfFileExists{bookmark.sty}{\usepackage{bookmark}}{\usepackage{hyperref}}
\hypersetup{
  pdftitle={Inference for categorical data},
  hidelinks,
  pdfcreator={LaTeX via pandoc}}
\urlstyle{same} % disable monospaced font for URLs
\usepackage[margin=1in]{geometry}
\usepackage{color}
\usepackage{fancyvrb}
\newcommand{\VerbBar}{|}
\newcommand{\VERB}{\Verb[commandchars=\\\{\}]}
\DefineVerbatimEnvironment{Highlighting}{Verbatim}{commandchars=\\\{\}}
% Add ',fontsize=\small' for more characters per line
\usepackage{framed}
\definecolor{shadecolor}{RGB}{248,248,248}
\newenvironment{Shaded}{\begin{snugshade}}{\end{snugshade}}
\newcommand{\AlertTok}[1]{\textcolor[rgb]{0.94,0.16,0.16}{#1}}
\newcommand{\AnnotationTok}[1]{\textcolor[rgb]{0.56,0.35,0.01}{\textbf{\textit{#1}}}}
\newcommand{\AttributeTok}[1]{\textcolor[rgb]{0.77,0.63,0.00}{#1}}
\newcommand{\BaseNTok}[1]{\textcolor[rgb]{0.00,0.00,0.81}{#1}}
\newcommand{\BuiltInTok}[1]{#1}
\newcommand{\CharTok}[1]{\textcolor[rgb]{0.31,0.60,0.02}{#1}}
\newcommand{\CommentTok}[1]{\textcolor[rgb]{0.56,0.35,0.01}{\textit{#1}}}
\newcommand{\CommentVarTok}[1]{\textcolor[rgb]{0.56,0.35,0.01}{\textbf{\textit{#1}}}}
\newcommand{\ConstantTok}[1]{\textcolor[rgb]{0.00,0.00,0.00}{#1}}
\newcommand{\ControlFlowTok}[1]{\textcolor[rgb]{0.13,0.29,0.53}{\textbf{#1}}}
\newcommand{\DataTypeTok}[1]{\textcolor[rgb]{0.13,0.29,0.53}{#1}}
\newcommand{\DecValTok}[1]{\textcolor[rgb]{0.00,0.00,0.81}{#1}}
\newcommand{\DocumentationTok}[1]{\textcolor[rgb]{0.56,0.35,0.01}{\textbf{\textit{#1}}}}
\newcommand{\ErrorTok}[1]{\textcolor[rgb]{0.64,0.00,0.00}{\textbf{#1}}}
\newcommand{\ExtensionTok}[1]{#1}
\newcommand{\FloatTok}[1]{\textcolor[rgb]{0.00,0.00,0.81}{#1}}
\newcommand{\FunctionTok}[1]{\textcolor[rgb]{0.00,0.00,0.00}{#1}}
\newcommand{\ImportTok}[1]{#1}
\newcommand{\InformationTok}[1]{\textcolor[rgb]{0.56,0.35,0.01}{\textbf{\textit{#1}}}}
\newcommand{\KeywordTok}[1]{\textcolor[rgb]{0.13,0.29,0.53}{\textbf{#1}}}
\newcommand{\NormalTok}[1]{#1}
\newcommand{\OperatorTok}[1]{\textcolor[rgb]{0.81,0.36,0.00}{\textbf{#1}}}
\newcommand{\OtherTok}[1]{\textcolor[rgb]{0.56,0.35,0.01}{#1}}
\newcommand{\PreprocessorTok}[1]{\textcolor[rgb]{0.56,0.35,0.01}{\textit{#1}}}
\newcommand{\RegionMarkerTok}[1]{#1}
\newcommand{\SpecialCharTok}[1]{\textcolor[rgb]{0.00,0.00,0.00}{#1}}
\newcommand{\SpecialStringTok}[1]{\textcolor[rgb]{0.31,0.60,0.02}{#1}}
\newcommand{\StringTok}[1]{\textcolor[rgb]{0.31,0.60,0.02}{#1}}
\newcommand{\VariableTok}[1]{\textcolor[rgb]{0.00,0.00,0.00}{#1}}
\newcommand{\VerbatimStringTok}[1]{\textcolor[rgb]{0.31,0.60,0.02}{#1}}
\newcommand{\WarningTok}[1]{\textcolor[rgb]{0.56,0.35,0.01}{\textbf{\textit{#1}}}}
\usepackage{graphicx}
\makeatletter
\def\maxwidth{\ifdim\Gin@nat@width>\linewidth\linewidth\else\Gin@nat@width\fi}
\def\maxheight{\ifdim\Gin@nat@height>\textheight\textheight\else\Gin@nat@height\fi}
\makeatother
% Scale images if necessary, so that they will not overflow the page
% margins by default, and it is still possible to overwrite the defaults
% using explicit options in \includegraphics[width, height, ...]{}
\setkeys{Gin}{width=\maxwidth,height=\maxheight,keepaspectratio}
% Set default figure placement to htbp
\makeatletter
\def\fps@figure{htbp}
\makeatother
\setlength{\emergencystretch}{3em} % prevent overfull lines
\providecommand{\tightlist}{%
  \setlength{\itemsep}{0pt}\setlength{\parskip}{0pt}}
\setcounter{secnumdepth}{-\maxdimen} % remove section numbering
\ifLuaTeX
  \usepackage{selnolig}  % disable illegal ligatures
\fi

\title{Inference for categorical data}
\author{}
\date{\vspace{-2.5em}}

\begin{document}
\maketitle

In August of 2012, news outlets ranging from the
\href{http://www.washingtonpost.com/national/on-faith/poll-shows-atheism-on-the-rise-in-the-us/2012/08/13/90020fd6-e57d-11e1-9739-eef99c5fb285_story.html}{Washington
Post} to the
\href{http://www.huffingtonpost.com/2012/08/14/atheism-rise-religiosity-decline-in-america_n_1777031.html}{Huffington
Post} ran a story about the rise of atheism in America. The source for
the story was a poll that asked people, ``Irrespective of whether you
attend a place of worship or not, would you say you are a religious
person, not a religious person or a convinced atheist?'' This type of
question, which asks people to classify themselves in one way or
another, is common in polling and generates categorical data. In this
lab we take a look at the atheism survey and explore what's at play when
making inference about population proportions using categorical data.

\hypertarget{the-survey}{%
\subsection{The survey}\label{the-survey}}

To access the press release for the poll, conducted by WIN-Gallup
International, click on the following link:

\emph{\url{https://www.scribd.com/document/136318147/Win-gallup-International-Global-Index-of-Religiosity-and-Atheism-2012}}

Take a moment to review the report then address the following questions.

\begin{enumerate}
\def\labelenumi{\arabic{enumi}.}
\tightlist
\item
  In the first paragraph, several key findings are reported. Do these
  percentages appear to be \emph{sample statistics} (derived from the
  data sample) or \emph{population parameters}?
\end{enumerate}

Answer: \emph{sample statistics}

\begin{enumerate}
\def\labelenumi{\arabic{enumi}.}
\setcounter{enumi}{1}
\tightlist
\item
  The title of the report is ``Global Index of Religiosity and
  Atheism''. To generalize the report's findings to the global human
  population, what must we assume about the sampling method? Does that
  seem like a reasonable assumption?
\end{enumerate}

Answer: In order to generalize the report's findings to the global human
population, we must assume the sampling method is \emph{random} and the
observations are \emph{independent}. It seems a reasonable assumption if
the survey is conduct correctly.

\hypertarget{the-data}{%
\subsection{The data}\label{the-data}}

Turn your attention to Table 6 (pages 15 and 16), which reports the
sample size and response percentages for all 57 countries. While this is
a useful format to summarize the data, we will base our analysis on the
original data set of individual responses to the survey. Load this data
set into R with the following command.

\begin{Shaded}
\begin{Highlighting}[]
\CommentTok{\#download.file("http://www.openintro.org/stat/data/atheism.RData", destfile = "atheism.RData")}
\CommentTok{\#load("atheism.RData")}

\FunctionTok{load}\NormalTok{(}\StringTok{\textquotesingle{}\textasciitilde{}/inf\_for\_categorical\_data/more/atheism.RData\textquotesingle{}}\NormalTok{)}
\end{Highlighting}
\end{Shaded}

\begin{enumerate}
\def\labelenumi{\arabic{enumi}.}
\setcounter{enumi}{2}
\tightlist
\item
  What does each row of Table 6 correspond to? What does each row of
  \texttt{atheism} correspond to?
\end{enumerate}

Answer: each row of Table 6 corresponds to proportions of each answer
group by countries, each row of \texttt{atheism} corresponds to answers
of each surveyed individuals

To investigate the link between these two ways of organizing this data,
take a look at the estimated proportion of atheists in the United
States. Towards the bottom of Table 6, we see that this is 5\%. We
should be able to come to the same number using the \texttt{atheism}
data.

\begin{enumerate}
\def\labelenumi{\arabic{enumi}.}
\setcounter{enumi}{3}
\tightlist
\item
  Using the command below, create a new dataframe called \texttt{us12}
  that contains only the rows in \texttt{atheism} associated with
  respondents to the 2012 survey from the United States. Next, calculate
  the proportion of atheist responses. Does it agree with the percentage
  in Table 6? If not, why?
\end{enumerate}

\begin{Shaded}
\begin{Highlighting}[]
\NormalTok{us12 }\OtherTok{\textless{}{-}} \FunctionTok{subset}\NormalTok{(atheism, nationality }\SpecialCharTok{==} \StringTok{"United States"} \SpecialCharTok{\&}\NormalTok{ year }\SpecialCharTok{==} \StringTok{"2012"}\NormalTok{)}
\FunctionTok{sum}\NormalTok{(us12}\SpecialCharTok{$}\NormalTok{response}\SpecialCharTok{==}\StringTok{\textquotesingle{}atheist\textquotesingle{}}\NormalTok{)}\SpecialCharTok{/}\FunctionTok{length}\NormalTok{(us12}\SpecialCharTok{$}\NormalTok{response)}
\end{Highlighting}
\end{Shaded}

Answer: proportion of atheist response is .0499 which agrees with the
percentage in Table 6

\hypertarget{inference-on-proportions}{%
\subsection{Inference on proportions}\label{inference-on-proportions}}

As was hinted at in Exercise 1, Table 6 provides \emph{statistics}, that
is, calculations made from the sample of 51,927 people. What we'd like,
though, is insight into the population \emph{parameters}. You answer the
question, ``What proportion of people in your sample reported being
atheists?'' with a statistic; while the question ``What proportion of
people on earth would report being atheists'' is answered with an
estimate of the parameter.

The inferential tools for estimating population proportion are analogous
to those used for means in the last chapter: the confidence interval and
the hypothesis test.

\begin{enumerate}
\def\labelenumi{\arabic{enumi}.}
\setcounter{enumi}{4}
\tightlist
\item
  Write out the conditions for inference to construct a 95\% confidence
  interval for the proportion of atheists in the United States in 2012.
  Are you confident all conditions are met?
\end{enumerate}

Answer: The sample's observations are independent, e.g.~are from a
simple random sample. At least 10 successes and 10 failures in the
sample: np = 50 \textgreater{} 10 and n(1 − p) = 952 ≥ 10, which
satisfied the success-failure condition.

If the conditions for inference are reasonable, we can either calculate
the standard error and construct the interval by hand, or allow the
\texttt{inference} function to do it for us.

\begin{Shaded}
\begin{Highlighting}[]
\FunctionTok{inference}\NormalTok{(us12}\SpecialCharTok{$}\NormalTok{response, }\AttributeTok{est =} \StringTok{"proportion"}\NormalTok{, }\AttributeTok{type =} \StringTok{"ci"}\NormalTok{, }\AttributeTok{method =} \StringTok{"theoretical"}\NormalTok{, }\AttributeTok{success =} \StringTok{"atheist"}\NormalTok{)}
\end{Highlighting}
\end{Shaded}

Note that since the goal is to construct an interval estimate for a
proportion, it's necessary to specify what constitutes a ``success'',
which here is a response of \texttt{"atheist"}.

Although formal confidence intervals and hypothesis tests don't show up
in the report, suggestions of inference appear at the bottom of page 7:
``In general, the error margin for surveys of this kind is \(\pm\) 3-5\%
at 95\% confidence''.

\begin{enumerate}
\def\labelenumi{\arabic{enumi}.}
\setcounter{enumi}{5}
\tightlist
\item
  Based on the R output, what is the margin of error for the estimate of
  the proportion of the proportion of atheists in US in 2012?
\end{enumerate}

Answer: Standard error = 0.0069 with 95 \% Confidence interval = (
0.0364 , 0.0634 )

\begin{enumerate}
\def\labelenumi{\arabic{enumi}.}
\setcounter{enumi}{6}
\tightlist
\item
  Using the \texttt{inference} function, calculate confidence intervals
  for the proportion of atheists in 2012 in two other countries of your
  choice, and report the associated margins of error. Be sure to note
  whether the conditions for inference are met. It may be helpful to
  create new data sets for each of the two countries first, and then use
  these data sets in the \texttt{inference} function to construct the
  confidence intervals.
\end{enumerate}

\hypertarget{how-does-the-proportion-affect-the-margin-of-error}{%
\subsection{How does the proportion affect the margin of
error?}\label{how-does-the-proportion-affect-the-margin-of-error}}

Imagine you've set out to survey 1000 people on two questions: are you
female? and are you left-handed? Since both of these sample proportions
were calculated from the same sample size, they should have the same
margin of error, right? Wrong! While the margin of error does change
with sample size, it is also affected by the proportion.

Think back to the formula for the standard error:
\(SE = \sqrt{p(1-p)/n}\). This is then used in the formula for the
margin of error for a 95\% confidence interval:
\(ME = 1.96\times SE = 1.96\times\sqrt{p(1-p)/n}\). Since the population
proportion \(p\) is in this \(ME\) formula, it should make sense that
the margin of error is in some way dependent on the population
proportion. We can visualize this relationship by creating a plot of
\(ME\) vs.~\(p\).

The first step is to make a vector \texttt{p} that is a sequence from 0
to 1 with each number separated by 0.01. We can then create a vector of
the margin of error (\texttt{me}) associated with each of these values
of \texttt{p} using the familiar approximate formula
(\(ME = 2 \times SE\)). Lastly, we plot the two vectors against each
other to reveal their relationship.

\begin{Shaded}
\begin{Highlighting}[]
\NormalTok{n }\OtherTok{\textless{}{-}} \DecValTok{1000}
\NormalTok{p }\OtherTok{\textless{}{-}} \FunctionTok{seq}\NormalTok{(}\DecValTok{0}\NormalTok{, }\DecValTok{1}\NormalTok{, }\FloatTok{0.01}\NormalTok{)}
\NormalTok{me }\OtherTok{\textless{}{-}} \DecValTok{2} \SpecialCharTok{*} \FunctionTok{sqrt}\NormalTok{(p }\SpecialCharTok{*}\NormalTok{ (}\DecValTok{1} \SpecialCharTok{{-}}\NormalTok{ p)}\SpecialCharTok{/}\NormalTok{n)}
\FunctionTok{plot}\NormalTok{(me }\SpecialCharTok{\textasciitilde{}}\NormalTok{ p, }\AttributeTok{ylab =} \StringTok{"Margin of Error"}\NormalTok{, }\AttributeTok{xlab =} \StringTok{"Population Proportion"}\NormalTok{)}
\end{Highlighting}
\end{Shaded}

\begin{enumerate}
\def\labelenumi{\arabic{enumi}.}
\setcounter{enumi}{7}
\tightlist
\item
  Describe the relationship between \texttt{p} and \texttt{me}.
\end{enumerate}

\hypertarget{success-failure-condition}{%
\subsection{Success-failure condition}\label{success-failure-condition}}

The textbook emphasizes that you must always check conditions before
making inference. For inference on proportions, the sample proportion
can be assumed to be nearly normal if it is based upon a random sample
of independent observations and if both \(np \geq 10\) and
\(n(1 - p) \geq 10\). This rule of thumb is easy enough to follow, but
it makes one wonder: what's so special about the number 10?

The short answer is: nothing. You could argue that we would be fine with
9 or that we really should be using 11. What is the ``best'' value for
such a rule of thumb is, at least to some degree, arbitrary. However,
when \(np\) and \(n(1-p)\) reaches 10 the sampling distribution is
sufficiently normal to use confidence intervals and hypothesis tests
that are based on that approximation.

We can investigate the interplay between \(n\) and \(p\) and the shape
of the sampling distribution by using simulations. To start off, we
simulate the process of drawing 5000 samples of size 1040 from a
population with a true atheist proportion of 0.1. For each of the 5000
samples we compute \(\hat{p}\) and then plot a histogram to visualize
their distribution.

\begin{Shaded}
\begin{Highlighting}[]
\NormalTok{p }\OtherTok{\textless{}{-}} \FloatTok{0.1}
\NormalTok{n }\OtherTok{\textless{}{-}} \DecValTok{1040}
\NormalTok{p\_hats }\OtherTok{\textless{}{-}} \FunctionTok{rep}\NormalTok{(}\DecValTok{0}\NormalTok{, }\DecValTok{5000}\NormalTok{)}

\ControlFlowTok{for}\NormalTok{(i }\ControlFlowTok{in} \DecValTok{1}\SpecialCharTok{:}\DecValTok{5000}\NormalTok{)\{}
\NormalTok{  samp }\OtherTok{\textless{}{-}} \FunctionTok{sample}\NormalTok{(}\FunctionTok{c}\NormalTok{(}\StringTok{"atheist"}\NormalTok{, }\StringTok{"non\_atheist"}\NormalTok{), n, }\AttributeTok{replace =} \ConstantTok{TRUE}\NormalTok{, }\AttributeTok{prob =} \FunctionTok{c}\NormalTok{(p, }\DecValTok{1}\SpecialCharTok{{-}}\NormalTok{p))}
\NormalTok{  p\_hats[i] }\OtherTok{\textless{}{-}} \FunctionTok{sum}\NormalTok{(samp }\SpecialCharTok{==} \StringTok{"atheist"}\NormalTok{)}\SpecialCharTok{/}\NormalTok{n}
\NormalTok{\}}

\FunctionTok{hist}\NormalTok{(p\_hats, }\AttributeTok{main =} \StringTok{"p = 0.1, n = 1040"}\NormalTok{, }\AttributeTok{xlim =} \FunctionTok{c}\NormalTok{(}\DecValTok{0}\NormalTok{, }\FloatTok{0.18}\NormalTok{))}
\end{Highlighting}
\end{Shaded}

These commands build up the sampling distribution of \(\hat{p}\) using
the familiar \texttt{for} loop. You can read the sampling procedure for
the first line of code inside the \texttt{for} loop as, ``take a sample
of size \(n\) with replacement from the choices of atheist and
non-atheist with probabilities \(p\) and \(1 - p\), respectively.'' The
second line in the loop says, ``calculate the proportion of atheists in
this sample and record this value.'' The loop allows us to repeat this
process 5,000 times to build a good representation of the sampling
distribution.

\begin{enumerate}
\def\labelenumi{\arabic{enumi}.}
\setcounter{enumi}{8}
\item
  Describe the sampling distribution of sample proportions at
  \(n = 1040\) and \(p = 0.1\). Be sure to note the center, spread, and
  shape.\\
  \emph{Hint:} Remember that R has functions such as \texttt{mean} to
  calculate summary statistics.
\item
  Repeat the above simulation three more times but with modified sample
  sizes and proportions: for \(n = 400\) and \(p = 0.1\), \(n = 1040\)
  and \(p = 0.02\), and \(n = 400\) and \(p = 0.02\). Plot all four
  histograms together by running the \texttt{par(mfrow\ =\ c(2,\ 2))}
  command before creating the histograms. You may need to expand the
  plot window to accommodate the larger two-by-two plot. Describe the
  three new sampling distributions. Based on these limited plots, how
  does \(n\) appear to affect the distribution of \(\hat{p}\)? How does
  \(p\) affect the sampling distribution?
\end{enumerate}

Once you're done, you can reset the layout of the plotting window by
using the command \texttt{par(mfrow\ =\ c(1,\ 1))} command or clicking
on ``Clear All'' above the plotting window (if using RStudio). Note that
the latter will get rid of all your previous plots.

\begin{enumerate}
\def\labelenumi{\arabic{enumi}.}
\setcounter{enumi}{10}
\tightlist
\item
  If you refer to Table 6, you'll find that Australia has a sample
  proportion of 0.1 on a sample size of 1040, and that Ecuador has a
  sample proportion of 0.02 on 400 subjects. Let's suppose for this
  exercise that these point estimates are actually the truth. Then given
  the shape of their respective sampling distributions, do you think it
  is sensible to proceed with inference and report margin of errors, as
  the reports does?
\end{enumerate}

\begin{center}\rule{0.5\linewidth}{0.5pt}\end{center}

\hypertarget{on-your-own}{%
\subsection{On your own}\label{on-your-own}}

The question of atheism was asked by WIN-Gallup International in a
similar survey that was conducted in 2005. (We assume here that sample
sizes have remained the same.) Table 4 on page 13 of the report
summarizes survey results from 2005 and 2012 for 39 countries.

\begin{itemize}
\item
  Answer the following two questions using the \texttt{inference}
  function. As always, write out the hypotheses for any tests you
  conduct and outline the status of the conditions for inference.

  \textbf{a.} Is there convincing evidence that Spain has seen a change
  in its atheism index between 2005 and 2012?\\
  \emph{Hint:} Create a new data set for respondents from Spain. Form
  confidence intervals for the true proportion of athiests in both
  years, and determine whether they overlap.

  \textbf{b.} Is there convincing evidence that the United States has
  seen a change in its atheism index between 2005 and 2012?
\item
  If in fact there has been no change in the atheism index in the
  countries listed in Table 4, in how many of those countries would you
  expect to detect a change (at a significance level of 0.05) simply by
  chance?\\
  \emph{Hint:} Look in the textbook index under Type 1 error.
\item
  Suppose you're hired by the local government to estimate the
  proportion of residents that attend a religious service on a weekly
  basis. According to the guidelines, the estimate must have a margin of
  error no greater than 1\% with 95\% confidence. You have no idea what
  to expect for \(p\). How many people would you have to sample to
  ensure that you are within the guidelines?\\
  \emph{Hint:} Refer to your plot of the relationship between \(p\) and
  margin of error. Do not use the data set to answer this question.
\end{itemize}

\leavevmode\vadjust pre{\hypertarget{license}{}}%
This is a product of OpenIntro that is released under a
\href{http://creativecommons.org/licenses/by-sa/3.0}{Creative Commons
Attribution-ShareAlike 3.0 Unported}. This lab was written for OpenIntro
by Andrew Bray and Mine Çetinkaya-Rundel.

\end{document}
